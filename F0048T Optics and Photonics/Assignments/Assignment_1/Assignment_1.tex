\documentclass{article}
\usepackage{amsmath}

\begin{document}
\title{F0048T: Datorövning: Plana vågor}
\author{Simon Johnsson \textit{johsim-7@student.ltu.se}}
\date{\today}
\maketitle

\section*{Uppgift 1}
  \[
    U(\bar{r})=e^{i(k\bar{s}\cdot\bar{r}+\alpha)}
  \]
  \subsection*{A - Beskriv vad som menas med vågtalet $k$, riktningsvektorn $\bar{s}$, positionsvektorn $|r|$ och faskonstanten $\alpha$.}
    \begin{itemize}
      \item Vågtalet $k$ används för att ge ekvationerna en enklare form, vågtalet representerar $k=\frac{2\pi}{\lambda}$
      \item Riktningsvektorn $\bar{s}$ är en vektor som beskriver i vilken riktning som vågen propagerar
      \item positionsvektorn $|\bar{r}|$ betecknar vilket avstånd $U$ mäts på från ljuskällan
      \item Faskonstanten $\alpha$ betecknar förskjutning av vågen i propageringsriktningen, $\bar{s}$.
    \end{itemize}
  \subsection*{B - Beskriv hur intensiteten beräknas från den komplexa amplituden och vad som menas med den komplexa amplitudens fas}
    Intensiteten för den komplexa amplituden beräknas med hjälp av
    \[
      I(\bar{r})=|U(\bar{r})|^2=U(\bar{r})U^*(\bar{r})
    \]
    Komplexa amplitudens fas innebär vågens relativa höjd jämfört med planet den propagerar i. Till exempel om vågen propagerar i $yz$-planet så är fasen värdet på $x$ i punkten fasen beräknas i.
  \subsection*{C - Skriv ner ekvationen för den komplexa amplituden och dess intensitet av en godtycklig plan monokromatisk våg med våglängden $\lambda=\lambda_0/n$ som propagerar i riktning $\bar{s}$ i ett fast kartesiskt koordinatsystem. Exemplifiera med ett fall då $\lambda=\lambda_0=1\mu m$ och $\bar{s}$ ligger i $yz$-planet med vinkeln $10^{\circ}$ mot $z$-axeln}
    De givna storheterna ger
    \begin{itemize}
      \item $k=\frac{2\pi}{\lambda}=\frac{2\pi n}{\lambda_0}$
      \item $\bar{s}=\begin{bmatrix}
        s_x\\
        s_y\\
        s_z
      \end{bmatrix}$
    \end{itemize}
    Detta ger komplexa amplituden
    \[
      U(\bar{r})=e^{i\big(\frac{2\pi n}{\lambda_0}\begin{bmatrix}
        s_x & s_y & s_z
      \end{bmatrix}^T\cdot\bar{r}+\alpha\big)}
    \]
    och intensiteten
    \[
      I(\bar{r})=e^{i\big(\frac{2\pi n}{\lambda_0}\begin{bmatrix}
        s_x & s_y & s_z
      \end{bmatrix}^T\cdot\bar{r}+\alpha\big)}e^{-i\big(\frac{2\pi n}{\lambda_0}\begin{bmatrix}
        s_x & s_y & s_z
      \end{bmatrix}^T\cdot\bar{r}+\alpha\big)}=1
    \]
    Med insättning av värdena som skulle exemplifieras med får vi:\\
    Komplexa amplituden
    \[
      U(\bar{r})=e^{i\big(2\pi 10^{12}\begin{bmatrix}
        0 & \sin{10^{\circ}} & \cos{10^{\circ}}
      \end{bmatrix}^T\cdot\bar{r}+\alpha\big)}
    \]
    och intensiteten kommer att fortsätta att vara $1$ då det endast är komplexa exponenter och vi inte ändrat någon amplitud.

\newpage
\section*{Uppgift 2}
  \subsection*{A - Beskriv vad som menas med fasmatchning. Skriv ner sambanden mellan de olika rikningscosinerna ($\{s_x,s_y,s_z\}$) vid gränsytan mellan de två mediumen.}

    Fasmatchning kommer ifrån när ljus bryts i ett medium så ändras riktningen samt våglängden, detta gör så att för att bibehålla kontinuerligteten i vågen så kan fasen på den utgående vågen behövas att fasförflyttas. \\

    För fasmatchning så används likheten att vågen i första mediet ska vara lika som vågen i det andra mediet i brytningsytan vilket ger ekvationen:
    \[
      \bar{f_1}\cdot\bar{r}=\bar{f_2}\cdot\bar{r}=\bar{f_3}\cdot\bar{r}
    \]
    Där $\bar{r}$ är en vektor i brytningsytan, $\bar{f_1}$, $\bar{f_2}$ och $\bar{f_3}$ är vektorer som beskriver den infallande, utfallande respektive reflekterade vågen. Givet av:
    \[
      \bar{f_1}=\frac{n_1}{\lambda_0}\bar{s_1},\quad\bar{f_1}=\frac{n_2}{\lambda_0}\bar{s_2},\quad\bar{f_1}=\frac{n_1}{\lambda_0}\bar{s_3}
    \]

    Där $\lambda_0$ är våglängden i vakum och de olika riktningscosinerna kan beskrivas om man lägger vågen i $yz$-planet genom:
    \[
      \bar{s_1} =
      \begin{bmatrix}
        s_{x1}\\s_{y1}\\s_{z1}
      \end{bmatrix}
      =
      \begin{bmatrix}
        0\\\sin{\theta_1}\\\cos{\theta_1}
      \end{bmatrix},
      \quad
      \bar{s_2} =
      \begin{bmatrix}
        s_{x2}\\s_{y2}\\s_{z2}
      \end{bmatrix}
      =
      \begin{bmatrix}
        0\\\sin{\theta_2}\\\cos{\theta_2}
      \end{bmatrix},
      \quad
      \bar{s_3} =
      \begin{bmatrix}
        s_{x3}\\s_{y3}\\s_{z3}
      \end{bmatrix}
      =
      \begin{bmatrix}
        0\\\sin{\theta_3}\\-\cos{\theta_3}
      \end{bmatrix}
    \]
    Där $s_1$ är den infallande riktningen, $s_2$ är den utfallande riktningen och $s_3$ är den reflekterade riktningen. $\theta_2=\arcsin{\big(\frac{n_1}{n_2}\sin{\theta_1}\big)}$ och $\theta_3=-\theta_1$.\\

    Vågen kan även behöva förflyttas i propageringsriktningen för att matcha i brytningsytan detta görs i så fall av att bestämma $\alpha$ före och efter brytningen så att vågorna matchar. Till exempel vid translation:
    \[
      e^{i(k_1\bar{s_1}\cdot\bar{r}+\alpha_1)}=e^{i(k_1\bar{s_2}\cdot\bar{r}+\alpha_2)}
    \]

  \subsection*{B - Skriv ner sambanden för Fresnels ekvationer för p-polariserat ljus och hur de förhåller sig till riktningscosinerna och brytningsindexen.}
    Fresnels ekvationer för de p-polariserade ljuset kan beskrivas med följande ekvationer:
    \[
      r_p = \frac{n_1s_{z2}-n_2s_{z1}}{n_1s_{z2}+n_2s_{z1}}
    \]
    \[
      t_p = \frac{2n_1s_{z1}}{n_1s_{z2}+n_2s_{z1}}
    \]
    Där $r_p$ och $t_p$ är reflektion- respektive transmissionskoefficienter.

  \subsection*{C - Beräkna samtliga riktningscosiner och transmissionskoefficienter ($t_p$) för fallen: \textbf{I} $\theta_1=20^{\circ}$, $n_1=1$, $n_2=1.6$ och \textbf{II} $\theta_1=20^{\circ}$, $n_1=1.6$, $n_2=1$}
    \textbf{FALL I:}\\
    $\theta_2\approx12.3429^{\circ}$\\
    Riktningscosinerna:
    \[
      \bar{s_1} =
      \begin{bmatrix}
        0\\0.3420\\0.9397
      \end{bmatrix}
      \qquad\qquad
      \bar{s_2} =
      \begin{bmatrix}
        0\\0.2138\\0.9769
      \end{bmatrix}
    \]
    Transmissionskoefficienten:
    \[
      t_p = \frac{2\times1\times0.9397}{1\times0.9769+1.6\times0.9397}\approx0.7577
    \]

    \textbf{FALL II:}\\
    $\theta_2\approx33.1773^{\circ}$\\
    Riktningscosinerna:
    \[
      \bar{s_1} =
      \begin{bmatrix}
        0\\0.3420\\0.9397
      \end{bmatrix}
      \qquad\qquad
      \bar{s_2} =
      \begin{bmatrix}
        0\\0.5472\\0.8370
      \end{bmatrix}
    \]
    Transmissionskoefficienten:
    \[
      t_p = \frac{2\times1.6\times0.9397}{1.6\times0.8370+1\times0.9397}\approx1.3195
    \]

\section*{Uppgift 3}
  \subsection*{Beräkna energitransmitansen $T$ för bägge fallen i uppgift 2c}
    \textbf{FALL I:}
    \[
      T=|t_p|^2\frac{n_2s_{z2}}{n_1s_{z1}}=|0.7577|^2\frac{1.6\times0.9769}{1\times0.9397}\approx0.9549
    \]
    \textbf{FALL II:}
    \[
      T=|t_p|^2\frac{n_2s_{z2}}{n_1s_{z1}}=|1.3195|^2\frac{1\times0.8370}{1.6\times0.9397}\approx0.9693
    \]

\newpage
\section*{Uppgift 4}
  \subsection*{A - Beskriv vad som menas med Brewstervinkeln, $theta_B$, och den kritiska vinkeln, $\theta_C$, samt hur dessa förhåller sig till brytningsindex och infallande vinklar.}
    Brewservinkeln är den vinkeln där allt ljus som reflekteras är polariserat och vinkeln bestämms utav ekvationen: $\theta_B=\arctan{\big(\frac{n_2}{n_1}\big)}$\\

    Kritiska vinkeln är en vinkel som när man har en vinkel som är större så transmiteras inget ljus, den betämms utav ekvationen: $\theta_C=\arcsin{\big(\frac{n_2}{n_1}\big)}$

  \subsection*{Beräkna Brewstervinkeln $\theta_B$ för fallet $n_1=1$, $n_2=1.6$. Beräkna den kritiska vinkeln $\theta_C$ för fallet $n_1=1.6$, $n_2=1$}
    Brewservinkeln:
    \[
      \theta_B=\arctan{\big(\frac{1.6}{1}\big)}=57.99
    \]
    Kritiska vinkeln:
    \[
      \theta_C=\arcsin{\big(\frac{1}{1.6}\big)}=38.68
    \]


\end{document}
