\documentclass{article}
\usepackage{amsmath}
\usepackage{amssymb}

\newcommand\varmp{\mathbin{\vcenter{\hbox{%
  \oalign{$\scriptstyle({+})$\cr
          \noalign{\kern-.3ex}
          \hfil$\scriptscriptstyle-$\hfil\cr}%
}}}}

\begin{document}
  \section{Vågoptik}
  %%%%%%%%%%%%%%%%%%%%%%%%%%%%%%%%%%%%%%%%%%%%%%%%%%%%%%%%%%%%%%%%%%%%%%%%%%%%%%%%%%%%%%%%%%%%%%%%%%%
  %%%%%%%%%%%%%%%%%%%%%%%%%%%%%%%%%%%%%%%%%%%%%%%%%%%%%%%%%%%%%%%%%%%%%%%%%%%%%%%%%%%%%%%%%%%%%%%%%%%
  %%%%%%%%%%%%%%%%%%%%%%%%%%%%%%%%%%%%%%%%%%%%%%%%%%%%%%%%%%%%%%%%%%%%%%%%%%%%%%%%%%%%%%%%%%%%%%%%%%%
  \subsection{Vågoptik, grunder}
  \begin{itemize}
    \item Reella vågfunktionen: $u(\bar{r},t)$
    \item Består av ljus som propagerar inom rymdvinkel $\Omega$
    \item Frekvenser inom spektrat $[\nu_{min},\nu_{max}]$
    \item $\nu\approx 10^{14}Hz$
    \item Vågekvationen: $\nabla^2u(\bar{r},t)-\frac{1}{c^2}\frac{\partial^2u(\bar{r},t)}{\partial t^2}=0$
    \item Fashastigheten: $c=c_0/n$
    \item Ljusetshastigeht: $c_0$
    \item Brytningsindex: $n$
  \end{itemize}

  \begin{equation}
    u(\bar{r},t)=Re{U(\bar{r},t)}=\frac{1}{2}[U(\bar{r},t)+U*(\bar{r},t)]
  \end{equation}
  $U(\bar{r},t)$ Komplex vågfunktion\\

  Generell lösning:
  \begin{equation}
    U(\bar{r},t)=\int_{\nu}\int_{\Omega}U(\bar{f},\nu)exp(i2\pi(\nu t-\bar{f}\cdot\bar{r}))d\bar{f}d{\nu}
  \end{equation}
  $U(\bar{f},\nu)$: Spektrala vågfunktionen\\

  \textbf{Mätbara storheter:}
  \begin{itemize}
    \item Area: $A_d$
    \item Integrationstid $T$
    \item Intensitet: $ I(\bar{r},t) = 2<u^2(\bar{r},t)>\qquad[W/m^2]$
    \item Effekt: $P(t) = \int_{A_d}I()\bar{r},t)dA\qquad[W]$
    \item Energi: $E = \int_TP(t)dt\qquad[J]$ \quad\textbf{Vilket är det som mäts!}
  \end{itemize}
  %%%%%%%%%%%%%%%%%%%%%%%%%%%%%%%%%%%%%%%%%%%%%%%%%%%%%%%%%%%%%%%%%%%%%%%%%%%%%%%%%%%%%%%%%%%%%%%%%%%
  %%%%%%%%%%%%%%%%%%%%%%%%%%%%%%%%%%%%%%%%%%%%%%%%%%%%%%%%%%%%%%%%%%%%%%%%%%%%%%%%%%%%%%%%%%%%%%%%%%%
  %%%%%%%%%%%%%%%%%%%%%%%%%%%%%%%%%%%%%%%%%%%%%%%%%%%%%%%%%%%%%%%%%%%%%%%%%%%%%%%%%%%%%%%%%%%%%%%%%%%
  \subsection{Monokromatiska vågor}
  Med hjälp av vågekvationen och vågfunktionen får man:
  \begin{equation}
    U(\bar{r},t)=U(\bar{r}e^{i2\pi\nu t})
  \end{equation}
  Rumsderiverad:
  \begin{equation}
    \nabla^2U(\bar{r},t)=\nabla^2\big(U(\bar{r})\big)e^{i2\pi\nu t}
  \end{equation}
  Tidsderiverad:
  \begin{equation}
    \frac{\partial^2U(\bar{r},t)}{\partial t^2} = -4\pi^2\nu^2U(\bar{r})e^{i2\pi\nu t}
  \end{equation}

  Vilket ger den \textbf{viktigaste} ekvationen inom optiken, Helmholz ekvation:
  \begin{equation}
    \mathbf{\bigg[\nabla^2U(\bar{r})+k^2U(\bar{r})\bigg]e^{i2\pi\nu t} = 0}
  \end{equation}
  Där:
  \begin{itemize}
    \item Frekvens: $\nu$
    \item Vågtal: $k=\frac{2\pi\nu}{c}=\frac{2\pi}{\lambda}$
    \item Dispersionsrelation: $c=\nu\lambda$
    \item Våglängd: $\lambda$
  \end{itemize}

  Då $e^{i2\pi\nu t}$ är fourierkerneln så kan man titta på den komplexa vågfunktionen i frekvensrummet, vid monokromatiskt ljus får vi bara en peak på var sida om noll. En för funktionen och en för komplexkonjugatet.\\

  Vid invers fouriertransform av $U(\bar{r},t)$ så får vi $u(\bar{r},t)=A(\bar{r})\cos{\phi(\bar{r})+s\pi\nu_0t}$\\

  \begin{itemize}
    \item Komplex Amplitud: $U(\bar{r})=A(\bar{r})e^{i\phi(\bar{r})}$
    \item Amplitud: $A(\bar{r})$
    \item Fas: $\phi(\bar{r})$
    \item Intensitet: $I(\bar{r})=|U(\bar{r})|^2=A^2(\bar{r})\qquad [W/m^2]$
    \item Effekt: $P = IA_d\qquad[W]$
    \item Emergitäthet: $W=\frac{I}{c}=\frac{E}{cA_dT}\qquad[J/m^3]$
    \begin{itemize}
      \item $A_d$ Detektorarea
      \item $T$ Exponeringstid
    \end{itemize}
  \end{itemize}

  \newpage
  \subsection{Monokromatiska sfäriska vågor}
    En sfärisk våg presenteras nästan på samma sätt som en plan våf förutom att amplituden ändras lite och den har en bestämd startpunkt.
    \[
      U(\bar{r})=\frac{A}{r}e^{-ikr}
    \]
    \begin{itemize}
      \item $\bar{r}=r\hat{r}$ \quad\textit{Vektor från centrum av den sfäriska vågen i riktning $\hat{r}$ och med längden $r$.}
      \item Amplituden: $A = A_0e^{i\phi_0}$
      \item Arean för en sfär: $A_s=4\pi r^2$
      \item Vågen har en konstant energi för varje $r$ vilket leder till: $I=\frac{E_0}{A_sT}=\frac{P_0}{4\pi r^2}$
    \end{itemize}
    Amplituden ges utav:
    \[
    A_o(\bar{r})=\sqrt{I}
    \]
    Paraxial våg:
    \[
      r = \sqrt{z^2+\rho^2}\approx z+\frac{\rho^2}{2z}
    \]
    \[
      U(\bar{r})\approx \frac{A}{z}e^{-ikz}e^{-ik\frac{\rho^2}{2z}=A(\bar{r})e^{-ikz}}
    \]
    Komplex amplitud:
    \[
      A(\bar{r})=\frac{A_0}{z}e^{-ik\frac{\rho^2}{2z}}
    \]
    Om man betraktar en liten del av en sfärisk våg som har en rymdvinkel $\Omega<<1$ på ett avstånd långt ifrån centrum så kommer det att i princip att vara en plan våg.
  \newpage
  \subsection{Avbildining med vågoptik}
    Avbildning kan till exempel vara att omvandla en utgående sfärisk våg till en anna ingående sfärisk våg. Eller att man omvandlar en sfärisk våg till en plan våg eller tvärtom. Dessa är med olika former av positiva linser.\\

    Om man använder sig utav en negativ lins så omvandlar man en sfärisk våg till en annan sfärisk våg med större krökning.\\

    Fasen av en sfärisk våg:
    \[
      e^{-ikr}
    \]
    där $r$ är avståndet till krökningscentrumet. Propagerar över linsytan:
    \[
     \phi=kr=ka\sqrt{1+\big(\frac{y}{a}\big)^2}\approx k\big(a+\frac{y^2}{2a}\big)
    \]
    Variationen efter linsytan ges utav
    \[
      \phi_y=k\frac{y^2}{2a}
    \]
    Om längden från vågcentrat till linsen är lika långt som linsens brännvidd så blir förändringen noll efter linsen
    \[
      a=f\Longrightarrow\phi_y=0\;\text{efter\;linsen}\Longrightarrow U_{lins}=e^{ik\frac{y^2}{2f}}
    \]
    Generellt:\\
    För att beskriva den komplexa amplituden efter linsen så multiplicerar man komplexa amplituden före linsen med linsens komplexa amplitud
    \[
      U^+=U^-U_{lins}\Longrightarrow U^+=A_0e^{-ik(a+\frac{y^2}{2a})}e^{ik\frac{y^2}{2f}}
    \]

    Där $U^-$ är optiska fältet före linsen och $U^+$ optiska fältet efter.\\

    Om man istället tittar på faserna så får vi
    \[
      \phi_y^+=\phi_y^--\phi_{lins}
    \]
    vilket man skulle kunna skriva om till
    \[
      \frac{1}{a}-\frac{1}{f}=\frac{1}{b}
    \]
    Vilket är linsformeln och $b$ blir den resulterande vågens centrum.\\

    Avbildning:
    \[
      U^+=A_0e^{-ik(a+\frac{y^2}{2a})}e^{ik\frac{y^2}{2f}}=A_0e^{-ik(a+\frac{y^2}{2b})}
    \]
    Där vi sätter $a=\alpha f$ där $\alpha>1$. Vilket skulle ge $b=-\frac{\alpha f}{\alpha -1}$. Detta betyder att den är en divergerande våg innan linsen som sedan blir en konvergerande våg efter linsen då $b$ är negativt, vilket betyder att krökningsradien hamnar framför vågen istället.\\

    Beräknar för en lupp istället:
    \[
      U^+=A_0e^{ik(a+\frac{y^2}{2a})}e^{ik\frac{y^2}{2f}}=A_0e^{-ik(a+\frac{y^2}{2b})}
    \]
    Där vi sätter $a=f$ istället, det gör att $\frac{1}{b}=0$ och utgående vågen blir en plan våg
    \[
      U^+=A_0e^{-ika}
    \]

    Avbildning från oändligheten kommer ge ett liknande resultat som föregående fast åt motsatt håll detta då $a\rightarrow\infty$ så ger linsformeln, $\frac{1}{b}=\frac{1}{a}-\frac{1}{f}$, att $b=-f$. Vilket gör att det blir ett helt omvänt system jämfört med föregående exempel och ekvationen blir
    \[
      U^+=A_0e^{-ik(a-\frac{y^2}{2f})}
    \]
    Om man istället tar ett exempel med en negativ lins så kommer $f=-|f|$ vilket leder till $b=\frac{a|f|}{|f|+a}$. Detta gör så att det kommer att vara divergerande vågor både före och efter linsen.

\newpage
\section{Stråloptik}
  \subsection{Gaussisk stråle}
    Sammansatt monokromatisk våg:
    \[
      U(\bar{r})=\int_{\pm NA}U(\bar{s})e^{-ik\bar{s}\cdot\bar{r}}d\bar{s}
    \]
    Gaussisk profil:
    \[
      U(\bar{s})=A_0\pi \omega_0^2e^{-\frac{s_x^2+s_y^2}{\sigma_s^2}}
    \]
    där $\bar{s}\approx x_x\hat{x}+s_y\hat{y}+\hat{z}$.\\
    Om man sätter ihop de två ovanstående ekvationerna så får man:
    \[
      \Rightarrow U(\bar{r})=A_0\pi\omega_0^2e^{-ikz}\int_{\pm NA}e^{-\frac{s_x^2+s_y^2}{\sigma_s^2}}e^{-ik(s_xx+s_yy)}ds_xds_y
    \]

    Lösning på formen:
    \[
      U(\bar{r})=A(\bar{r})e^{-ikz}
    \]
    \[
      A(\bar{r})=\frac{A_1}{q(z)}e^{-ik\frac{\rho^2}{2q(z)}};\quad\rho^2=x^+y^2
    \]
    Strål parameter:
    \[
      q(z)=z+iz_0
    \]
    $z_0$ definerar strålens samtliga egenskaper. \\

    Utveckla $q(z)$:
    \[
      A(\bar{r})=A_0\frac{\omega_0}{\omega(z)}e^{-\frac{\rho^2}{\omega^2(z)}}e^{-ik\frac{\rho^2}{2R(z)}+i\rho(z)}
    \]
    \[\begin{cases}
      \omega(z)=\omega_0\sqrt{1+\big(\frac{z}{z_0}\big)^2};&\text{Radie}\\
      R(z)=z\bigg(1+\big(\frac{z_0}{z}\big)^2\bigg);&\text{Faskrökning}\\
      \rho(z)=\arctan{\frac{z}{z_0}};&\text{Fasskift}\\
      \omega_0=\sqrt{\frac{\lambda z_0}{\pi}};&\text{Midjeradie}\\
      \theta_0=\sigma_s=NA=\frac{\omega_0}{z_0}=\frac{\lambda}{\pi\omega_0};&\text{Divergensvinkel}
    \end{cases}\]
    Notera sambanden:
    \[
      z_0=\frac{\pi\omega_0}{\lambda}\Rightarrow\text{Smal midja, kort skärpedjup}
    \]
    \[
      \theta_0=\frac{\omega_0}{z_0}=\frac{\lambda}{\pi\omega_0}\Rightarrow\text{Smal midja, stor divergens}
    \]
    $\therefore$ smal midja $\Rightarrow$ \textit{"Snabbt sfärisk vågtyp"}\\

    Intensitetsfördelningen:
    \[
      I(\rho,z)=\frac{2\rho}{\pi\omega^2(z)}e^{-\frac{2\rho^2}{\omega^2(z)}}
    \]

  \subsection{Mainpulation av gaussisik stråle}
    Vid avbildning av en stråle med midja i $z_1=0$ genom en lins och sedan ihop igen i $z_2=0$ (fokus). På vänstra sidan beskrivs strålen med $z_{01}$ och på högra sidan $z_{02}$. Randvilkoret säger att båda strålarna ska ha samma bredd i linsen. Detta kan bskrivas med:
    \[
      \textbf{R.V.}\qquad\omega_1(L_1)=\omega_2(-L_2)
    \]
    \[
      \begin{cases}
        \omega_1(L_1)=\omega_{01}\sqrt{1+\big(\frac{L_1}{z_{01}}\big)^2};&\omega_{01}=\sqrt{\frac{\lambda z_{01}}{\pi}}\\
        \omega_2(-L_2)=\omega_{02}\sqrt{1+\big(\frac{L_2}{z_{02}}\big)^2};&\omega_{02}=\sqrt{\frac{\lambda z_{02}}{\pi}}
      \end{cases}
    \]
    Faskrökningen över linsen:
    \[
      \phi_{1-2}=-k\frac{\rho^2}{2R_1(L_1)}+k\frac{\rho^2}{2f}=-k\frac{\rho^2}{2R_2(-L_1)}
    \]
    Vilket ger:
    \[
      \frac{1}{f}=\frac{1}{R_1(L_1)}-\frac{1}{R_2(-L_2)};\qquad\textit{("Linsformeln")}
    \]
    Generellt sätt att behandla avbildning och manipulation av gaussisk stråle är sambandet:
    \[
      q_2=\frac{Aq_1+B}{Cq_1+D};\qquad M=
      \begin{bmatrix}
        A&B\\
        C&D
      \end{bmatrix}
    \]
    \textbf{Exempel:} Translation en sträcka $z=d$ från fokus.
    \[
      M=
      \begin{bmatrix}
        1&d\\
        0&1
      \end{bmatrix};\qquad q_1=iz_0
    \]
    \[
      q_2=\frac{iz_0+d}{1}=d+iz_0
    \]
  \subsection{Integralen av en exponentialfunktion}
    Integralen: $g(a,b,c)=\int_{-\infty}^{\infty}Ae^{-ax^2+bx+c}dx$ har lösningen:
    \[
      g(a,b,c)=A\sqrt{\frac{\pi}{a}}e^{\frac{b^2}{4a}+c}
    \]
    \textbf{Exempel:} Gaussisk puls.
    \[
      \begin{cases}
        U(z,t)=\int A(\nu)e^{i2\pi \nu\tau}d\nu\\
        A(\nu)=A_0e^{-\frac{(\nu\mp\nu_0)^2}{\sigma_{\nu}^2}}
        \tau=t-\frac{z}{c}
      \end{cases}
    \]
    Identifikation och variabelbyte: $\zeta=\nu-\nu_0$
    \[
      A_0\int e^{-\frac{\zeta^2}{\sigma_{\nu}^2}}e^{i2\pi\zeta\tau}e^{i2\pi\nu_0\tau}d\zeta
    \]
    \[
      \rightarrow a=\frac{1}{\sigma_{\nu}^2};\quad b=i2\pi\tau;\quad c=i2\pi\nu_0\tau
    \]
    \[
      U(z,t)=A_0\sigma_{\nu}\sqrt{\pi}e^{-\frac{\tau^2}{\sigma_{\tau}^2}}e^{i2\pi\nu_0\tau};\quad\sigma_{\tau}=\frac{1}{\pi\sigma_{\nu}}
    \]
    Integralen för 2-dimensioner:
    \[
      g(a,b_x,b_y,c)=\int_{-\infty}^{\infty}\int_{-\infty}^{\infty} Ae^{(-a(x^2+y^2)+b_xx+b_yy+c)}dxdy
    \]
    Har lösningen:
    \[
      a(a,b_x,b_y,c)=A\frac{\pi}{a}e^{\frac{b_x^2+b_y^2}{4a}+c}
    \]




\end{document}
