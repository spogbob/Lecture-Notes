\documentclass{article}
\usepackage{amsmath}
\usepackage[swedish]{babel}

\begin{document}
  \section*{Uppgift 1}
    \textbf{Givet:} Proton med kinetisk energi på $E_k=6.50keV$ samt att vi vet att massan för en proton är $m_p=938.272MeV/c^2$

    \subsection*{A - Vad är protonens linjära momentum?}
      Det linjära momentumet för en proton ges utav ekvationen:
      \begin{equation}
        p=m_pv
        \label{eq:momentum}
      \end{equation}
      Den kinetiska energin kan beräknas med hjälp utav ekvation:
      \begin{equation}
        E_k=\frac{1}{2}m_pv^2
        \label{eq:kinetic}
      \end{equation}
      Med hjälp av ekvation (\ref{eq:momentum}) och (\ref{eq:kinetic}) så kan en ekvation för $p$ göras som endast innehåller kända värden:
      \begin{equation}
        p=\sqrt{2E_km_p}
      \end{equation}
      Nummeriskt så blir det:
      \begin{equation}
        p=\sqrt{2\times6.50\times10^3\times938.272\times10^6}eV/c\approx\mathbf{2.47MeV/c}
      \end{equation}

    \subsection*{B - Vad är protonens de Broglie våglängd?}
      De Broglie våglängden ges utav ekvationen:
      \begin{equation}
        \lambda=\frac{h}{p}
      \end{equation}
      Där $h$ är Plancks konstant med ett värde på $4.1356673feVs$ och $p$ är momentumet från föregående del.
      \begin{equation}
        \lambda=\frac{4.1356673\times10^{-15}eVs}{2.469568\times10^{6}eV/c}=1.67465187\times10^{-21}cs
      \end{equation}
      För att skriva om våglängden till SI-enheter så multipliceras svaret med ljusets hastihet, $c=299792458m/s$:
      \begin{equation}
        \lambda=1.67465187\times10^{-21}s\times299792458m/s\approx\mathbf{0.502pm}
      \end{equation}

  \newpage
  \section*{Uppgift 2}
      \textbf{Givet:} Protonen rör sig i en endimensionell låda med längden $a=1.50nm$ med potentialen:
      \[
        V(x)=
        \begin{cases}
          0,      & \text{för}\quad0\leq x\leq a\\
          \infty, & \text{annars}
        \end{cases}
      \]
      Givet är även att då tiden $t=0$ så är vågfunktionen $\psi(x)$ är partikeln kända att vara:
      \[
        \psi(x)=
        \begin{cases}
          Ax^2(a-x),& \text{för}\quad0\leq x\leq a\\
          0,        & \text{annars}
        \end{cases}
      \]
      Där $A$ är en konstant
      \subsection*{A - Bestäm konstanten A så att vågfunktionen är normaliserad}
        Normalisering av vågekvationen i en endimensionell låda bestämms med ekvation:
        \[
          1=\int_{0}^{a}|\psi(x)|^2dx
        \]
        Med insättning av $\psi(x)$ blir ekvationen:
        \[
          1=\int_{0}^{a}|Ax^2(a-x)|^2dx=|A|^2\int_0^ax^4(a-x)^2dx=|A|^2\int_0^a(a^2x^4-2ax^5+x^6)dx=
        \]
        \[
          =|A|^2\bigg[\frac{a^2x^5}{5}-\frac{ax^6}{3}+\frac{x^7}{7}\bigg]_0^a=|A|^2\bigg(\frac{a^7}{5}-\frac{a^7}{3}+\frac{a^7}{7}\bigg)=|A|^2\frac{a^7}{105}
        \]
        $A$ bryts ut ur ekvationen samt värdet på $a$ läggs in:
        \[
          A=\sqrt{\frac{105}{a^7}}=\sqrt{\frac{105}{(1.50\times10^{-9})^7}}=\mathbf{264575.1311}
        \]
      \subsection*{B - Bestäm sannolikheten att det är grundtillståndets energi om man mäter energin}
        Sannolikheten att det är en specifik energinivå ges utav $|c_n|^2$ där $c_n$ ges utav:
        \[
          c_n=\int_0^a\psi_n^*f(x)dx
        \]
        Där $\psi_n^*$ ges utav:
        \[
          \psi_n^*=\psi_n=\sqrt{\frac{2}{a}}\sin{\frac{n\pi x}{a}}
        \]
        och $f(x)$ ges utav:
        \[
          f(x)=Ax^2(a-x)^2
        \]
        $c_n$ blir då:
        \[
          c_n=\sqrt{\frac{2}{a}}A\int_0^a\sin{\frac{n\pi x}{a}}x^2(a-x)^2
        \]
        Sätter in algebraiska versionen av $A$ innan samt förenklar $c_n$:
        \[
          c_n=\sqrt{\frac{2}{a}}\sqrt{\frac{105}{a^7}}\int_0^a\sin{\frac{n\pi x}{a}}x^2(a-x)dx=\sqrt{210}a^{-4}\int_0^a\sin{\frac{n\pi x}{a}}x^2(a-x)dx=
        \]
        \[
          =\sqrt{210}a^{-4}\int_0^a\sin{\frac{n\pi x}{a}}(ax^2-x^3)dx=
        \]
        \[
          =\sqrt{210}a^{-4}\bigg(a\int_0^a\sin{\frac{n\pi x}{a}}x^2dx-\int_0^a\sin{\frac{n\pi x}{a}}x^3dx\bigg)
        \]
        Beräknar integralerna var för sig för att underlätta:
        \[
          \int_0^a\sin{\frac{n\pi x}{a}}x^2dx=\bigg[-\frac{a\cos{\frac{n\pi x}{a}}}{n\pi}x^2\bigg]_0^a-\int_0^a-\frac{a\cos{\frac{n\pi x}{a}}}{n\pi}2xdx=
        \]
        \[
          =-\frac{a^3(-1)^n}{n\pi}+\bigg[\frac{2a^2\sin{\frac{n\pi x}{a}}}{n^2\pi^2}x\bigg]_0^a-\int_0^a\frac{2a^2\sin{\frac{n\pi x}{a}}}{n^2\pi^2}dx=
        \]
        \[
          =-\frac{a^3(-1)^n}{n\pi}-\bigg[-\frac{2a^3\cos{\frac{n\pi x}{a}}}{n^3\pi^3}\bigg]_0^a=\frac{a^3((-1)^n(2-n^2\pi^2)-2)}{n^3\pi^3}
        \]
        Och andra integralen:
        \[
          \int_0^a\sin{\frac{n\pi x}{a}}x^3dx=\bigg[-\frac{a\cos{\frac{n\pi x}{a}}}{n\pi}x^3\bigg]_0^a-\int_0^a-\frac{a\cos{\frac{n\pi x}{a}}}{n\pi}3x^2dx=
        \]
        \[
          =-\frac{a^4(-1)^n}{n\pi}+\bigg[\frac{3a^2\sin{\frac{n\pi x}{a}}}{n^2\pi^2}x^2\bigg]_0^a-\int_0^a\frac{6a^2\sin{\frac{n\pi x}{a}}}{n^2\pi^2}xdx=
        \]
        \[
          =-\frac{a^4(-1)^n}{n\pi}-\bigg[-\frac{6a^3\cos{\frac{n\pi x}{a}}}{n^3\pi^3}x\bigg]_0^a+\int_0^a-\frac{6a^3\cos{\frac{n\pi x}{a}}}{n^3\pi^3}dx=
        \]
        \[
          =-\frac{a^4(-1)^n}{n\pi}+\frac{6a^4(-1)^n}{n^3\pi^3}-\bigg[\frac{6a^4\sin{\frac{n\pi x}{a}}}{n^4\pi^4}\bigg]_0^a=\frac{a^4(-1)^n(6-n^2\pi^2)}{n^3\pi^3}
        \]
        Sätt in de båda integralerna igen:
        \[
          c_n=\sqrt{210}a^{-4}\bigg(a\frac{a^3((-1)^n(2-n^2\pi^2)-2)}{n^3\pi^3}-\frac{a^4(-1)^n(6-n^2\pi^2)}{n^3\pi^3}\bigg)=
        \]
        \[
          =\frac{\sqrt{210}}{n^3\pi^3}\bigg(\big((-1)^n(2-n^2\pi^2)-2\big)-(-1)^n(6-n^2\pi^2)\bigg)=\frac{2\sqrt{210}}{n^3\pi^3}\bigg(2(-1)^{n+1}-1\bigg)
        \]
        För att bräkna sannolikheten för att det ska vara grundtillståndets energi om man mäter energin så sättes $n=1$ in i $|c_n|^2$:
        \[
          |c_n|^2=\bigg(\frac{2\sqrt{210}}{\pi^3}\big(2(-1)^{2}-1\big)\bigg)^2=0.873736...\approx87.4\%
        \]



\end{document}
