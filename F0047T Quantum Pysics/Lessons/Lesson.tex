\documentclass{article}
\usepackage{amsmath}
\begin{document}
%%%%%%%%%%%%%%%%%%%%%%%%%%%%%%%%%%%%%%%%%%%%%%%%%%%%%%%%%%%%%%%%%%%%%
%                               Title                               %
%%%%%%%%%%%%%%%%%%%%%%%%%%%%%%%%%%%%%%%%%%%%%%%%%%%%%%%%%%%%%%%%%%%%%
  \title{Kvantfysik F0047T}
  \author{Simon Johnsson}
  \date{Luleå -- HT20}
  \maketitle

%%%%%%%%%%%%%%%%%%%%%%%%%%%%%%%%%%%%%%%%%%%%%%%%%%%%%%%%%%%%%%%%%%%%%
%                             Lesson 6                              %
%%%%%%%%%%%%%%%%%%%%%%%%%%%%%%%%%%%%%%%%%%%%%%%%%%%%%%%%%%%%%%%%%%%%%
  \textbf{Från och med lektion 6!!!}
  \section{Potential som elastisk potentiell energi}
  I klassisk mekanik är elastisk potentiell energi $\frac{1}{2}kx^2$ men inte riktigt i QM.\\

  Går att lösa på två sätt:
  \begin{enumerate}
    \item Algebraisk metod
    \item Analytisk metod
  \end{enumerate}

  \subsection{Algebraisk metod}
  Tar ut algebraiskt med hjälp av TOSE stegoperatorerna $\hat{a}_+$ och $\hat{a}_-$. Stegoperatorerna är ej kommutativa så vi kan inte bara byta plats på dem.

  Vi kan ta fram en kommutator för lägesoperatorn och rörelsemängdsoperatorn $[\hat{x},\hat{p}]$:\\
  Vi använder oss av en testfunktion $f(x)$ för att kunna evaluera Stegoperatorerna.

  \[
    \begin{split}
      [\hat{x},\hat{p}]f(x)=(\hat{x}\hat{p}-\hat{p}\hat{x})f(x)=-i\hbar\bigg(x\frac{d}{dx}f(x)-\frac{d}{dx}(xf(x))\bigg)=\\
      -i\hbar\bigg[x\frac{d}{dx}f(x)-\frac{dx}{dx}f(x)-x\frac{d}{dx}f(x)\bigg]=i\hbar f(x)
    \end{split}
  \]

  $[\hat{x},\hat{p}]=i\hbar$ kallas "canonical commutation relation"/"Osäkerhets sammbandet"\\

  Detta resultat appliceras på stegoperatorerna:
  \[
    \hat{a}_-\hat{a_+}=\frac{1}{2\hbar m \omega}[\hat{p}^2+(m\omega\hat{x})^2]-\frac{i}{2\hbar}[\bar{x},\bar{p}]
  \]
\[
  \hat{a}_-\hat{a_+}=\frac{1}{\hbar\omega}\hat{H}-\frac{i}{2\hbar}(i\hbar)=\frac{1}{\hbar\omega}\bar{H}+\frac{1}{2}
\]

\[
  \hat{H}=\hbar(\hat{a}_-\hat{a}_+-\frac{1}{2})
\]
Jämföra med att byta plats på stegoperatorerna:
\[
  \hat{a}_+\hat{a}_-=\frac{1}{\hbar\omega}\bar{H}-\frac{1}{2}
\]

\[
  \hat{H}=\hbar(\hat{a}_+\hat{a}_-+\frac{1}{2})
\]

Vilker leder till att vi kan skriva hamiltonoperatorn $\hat{H}=\hbar\omega(\hat{a}_{\mp}\hat{a}_{\pm}\mp\frac{1}{2})$\\

Sätter in uttrycket i SE:
\[
  \hbar\omega(\hat{a}_{\mp}\hat{a}_{\pm}\mp\frac{1}{2})\Psi=E\Psi
\]

Om vågfunktionen $\Psi$ är en egenfunktion med egenvärde $E$, vad är $\hat{a}_{\pm}\Psi$?

Från SE för harmonisk oscillator:\quad $\hat{H}\hat{a}_+\hat{a}_-+\frac{1}{2}=\hbar\omega(\hat{a}_+\hat{a}_-+\frac{1}{2})\hat{a}_+\Psi$

\[
  \hat{H}(\hat{a}_+,\Psi)=\hbar\omega(\hat{a}_+\hat{a}_-\hat{a}_++\frac{1}{2}\hat{a}_+)\Psi=\hbar\omega\hat{a}_+(\hat{a}_-\hat{a}_++\frac{1}{2})\Psi
\]

För att kommutatorn $[\hat{a}_-,\hat{a}_+]$ ska vara kommutativt så måste den vara lika med $1$ detta leder till:

\[
  \hat{H}(\hat{a}_+\Psi)=\hbar\omega\hat{a}_+(1+\hat{a}_+\hat{a}_-+\frac{1}{2})\Phi=\hat{a}_+(\hat{H}+\hbar\omega)\Psi=\hat{a}_+(E+\hbar\omega)\Psi
\]

Mellantermen är en konstant och kommutativ med de andra termerna:
\[
  \hat{H}(\hat{a}_+\Psi)=(E+\hbar\omega)\hat{a}_+\Psi
\]

Detta betyder att om $\hat{H}\Psi=E\Psi$ så är $\hat{a}_+\Psi$ en egenfunktion med egenvärde $E+\hbar\omega$.\\
På samma sätt så ger det att $\hat{a}_-$ är en egenfunktion med egenvärde $E-\hbar\omega$.
\[
  \hat{H}(\hat{a}_-\Psi)=(E-\hbar\omega)\hat{a}_-\Psi
\]

Detta gör så att om man multiplicerar vågfunktionen med steg-upp-operatorn $\hat{a}_+$ så ökar energin med $\hbar\omega$ för varje gång man gör det. Och på samma sätt för steg-ner-operatorn men denna kan inte gå oändligt långt ner utan det finns ett stopp nedåt på $E_0$.

\textbf{OBS:} $\hat{a}_{\pm}\Psi$ behöver inte vara normerade även om $\Psi$ är det.

Vi tar lägsta energinivån $E_0$, om vi applicerar $\hat{a}_-$: $\hat{a}_-\Psi_0=0$\\

I SE:
\[
  \frac{d}{dx}\Psi_0+\frac{m\omega}{\hbar}x\Psi_0=0
\]

Om man integrerar detta får man:
\[
  \ln{\Psi_0}=-\frac{m\omega}{\hbar}\frac{x^2}{2}+c
\]
vilket kan skrivas om till:
\[
  \Psi_0=Ae^{-\frac{m\omega}{2\hbar}x^2}
\]

För allmänn lösning så behöver vi först normera $\Psi_0$
\[
  1=\int_{-\infty}^{+\infty}|A|^2e^{-(m\omega x^2)/\hbar}dx=A^2\sqrt{\frac{\pi\hbar}{m\omega}}
\]
\[
  A =\sqrt[4]{\frac{m\omega}{\pi\hbar}}
\]
Vilket ger:
\[
  \Psi_0=\sqrt[4]{\frac{m\omega}{\pi\hbar}}e^{-\frac{m\omega}{2\hbar}x^2}
\]
Detta är stationära tillståndet för harmoniska oscillatorn för lägsta energi.\\

Hur ser $E_0$ ut?\\
Från SE: Beräkna energi för $\Psi_0$:
\[
  \hat{H}=\hbar\omega(\hat{a}_+\hat{a}_-+\frac{1}{2})=\frac{1}{2}\hbar\omega\Psi_0=E_0\Psi_0
\]
\[
  E_0=\frac{1}{2}\hbar\omega
\]

$\Psi_0$, $E_0$ är grundtillståndet vilket gör att de resterande är exiterade tillstånd och kan beskrivas med hjälp av att kliva uppåt med $\hat{a}_+$:\\

Energi för exiterade tillstånd:
\[
  E_n=(n+\frac{1}{2})\hbar \omega
\]

vågfunktionen för exiterade tillstånd:
\[
  \Psi_n(x)=A_n(\hat{a}_+)^n\Phi_0(x)
\]
Där $n$ är större eller lika med $0$.

Den generella bågfunktionen skulle då bli:
\[
  \Psi(x,t)=\sum_{n=0}^{\infty}c_n\Psi_n(x,t)
\]

\textbf{Titta själv på sida 45-46 och räkna på det!}
%%%%%%%%%%%%%%%%%%%%%%%%%%%%%%%%%%%%%%%%%%%%%%%%%%%%%%%%%%%%%%%%%%%%%
%                             Lesson 6                              %
%%%%%%%%%%%%%%%%%%%%%%%%%%%%%%%%%%%%%%%%%%%%%%%%%%%%%%%%%%%%%%%%%%%%%
  \newpage
  \subsection{Analytisk metod}
    \textbf{OBS:} \textit{Denna metod kan användas till många andra potentialer så den är mer generell än den algebraiska metoden}\\

    Lösning till SE:
    \[
      -\frac{\hbar^2}{2m}\frac{d^2\Psi}{dx^2}+\frac{1}{2}m\omega^2x^2\Psi=E\Psi
    \]
    Föra över till dimensionsfri \textit{(multiplicerar med $\frac{2}{\hbar\omega}$)}:
    \[
      -\frac{d^2\Psi}{d\zeta^2}+\zeta^2\Psi=K\Psi
    \]
    Löser vi SE så får vi $K->0$ och $\zeta->\infty$ vilket ger:
    \[
      \Psi(\zeta)=Ae^{-\frac{\zeta^2}{2}}+Be^{\frac{\zeta^2}{2}}
    \]
    Här är första termen normerbar men inte andra då om $\zeta->\infty$ så går hela termen mot $\infty$.
    \[
      \Psi(\zeta)=h(\zeta)e^{-\frac{\zeta^2}{2}}\qquad\text{Fysikaliska Lösningen}
    \]
    Detta är fortfarande en exakt lösning.\\

    Hur ser SE ut nu?
    \[
      \frac{d^2\Psi}{d\zeta^2}+(K-\zeta^2)\Psi=0
    \]
    Vilket när man beräknat dubbelderivatan kan skrivas som:
    \[
      \frac{d^2h(\zeta)}{d\zeta^2}-2\zeta\frac{dh(\zeta)}{d\zeta}+(K-1)h(\zeta)=0
    \]
    \begin{itemize}
      \item För små $\zeta$ ansätts en \underline{potensserie}:\quad$h(\zeta)=\sum_{m=0}^{\infty}a_m\zeta^m$\\
      Detta ger SE:
      \[
        \sum_{m=0}^{\infty}a_{m+2}(m+2)(m+1)\zeta^m-2\sum_{m=0}^{\infty}a_mm\zeta^m+(K-1)\sum_{m=0}^{\infty}a_m\zeta^m=0
      \]
      Potenser för $\zeta^m\Rightarrow a_{m+2}(m+2)(m+1)+(K-1-2m)a_m=0$\\

      Hur ser serien ut?
      \begin{itemize}
        \item För stora $\mathbf{m}$ skulle serien vara propotionerlig med $\zeta^2e^{\zeta^2}$ vilken inte är normerbar. D.v.s. att serien måste ha ett högsta $m$-värde.
      \end{itemize}
      \[
        (m+2)(m+1)a_{m+2}=(2m+1-K)a_m
      \]
      Om $K=2n+1$ då stannar serien ($K=\frac{2E}{\hbar\omega}$) vilket gör $\frac{2E}{\hbar\omega}=(2n+1)$ vilket ger:
      \[
        E=(n+\frac{1}{2})\hbar \omega
      \]
      Och detta är samma ekvation som i den algebraiska lösningen.
    \end{itemize}
    $h(\zeta)$ kallas för \textbf{Hermite polynom $H(\zeta)$}\\
    SE \textit{(med $H(\zeta)$)}:
    \[
      \frac{d^2H_n(\zeta)}{d\zeta^2}-2\zeta\frac{dH_n(\zeta)}{d\zeta}+2nH_n(\zeta)=0
    \]
    Där $\Psi_n(x)$ bestämms utav:
    \[
      \Psi_n(x)=\sqrt[4]{\frac{m\omega}{\pi\hbar}}\frac{1}{\sqrt{2^nn!}}H_n(\zeta)e^{-\frac{\zeta^2}{2}}
    \]
\newpage
\section{Potential $V(x)=0\Rightarrow$ den fria partikeln}
  SE: $-\frac{\hbar^2}{2m}\frac{d^2\Psi}{dx^2}=E\Psi$
  \[
    \frac{d^2\Psi}{dx^2}+\frac{E2m}{\hbar^2}\Psi=0
  \]
  Partikeln kan ha alla energier $(E>0)$.\\
  Lösningen:
  \[
    \Psi(x)=Ae^{ikx}+Be^{-ikx}
  \]
  Allmänna lösningen:
  \[
    \Psi(x,t)=Ae^{ikx}e^{-i\frac{E}{\hbar}t}+Be^{-ikx}e^{-i\frac{E}{\hbar}t}
  \]
  Med $\frac{E2m}{\hbar^2}=k^2\quad\Rightarrow\quad E=\frac{\hbar^2k^2}{2m}$
  \[
    \pm ikx-i\frac{\hbar k^2}{2m}t=\pm ik\big(s\mp\frac{\hbar k}{2m}t\big)
  \]
  \[
    \Psi(x,t)=Ae^{ik\big(x-\frac{\hbar k}{2m}t\big)}+Be^{-ik\big(x+\frac{\hbar k}{2m}t\big)}
  \]
  Där första termen är en våg som går åt höger och andra en våg som går åt vänster.\\
  Skriver vi $\Psi(x,t)$ mer kompakt:
  \[
    \Psi_k(x,t)=Ae^{i\big(kx-\frac{\hbar k^2}{2m}t\big)}
  \]
  Där $k=\pm\frac{\sqrt{2mE}}{\hbar}$\\

  Vågnummer:\quad $k=\frac{2\pi}{\lambda}$\\
  Våglängd:\quad $\lambda=\frac{2\pi}{|k|}$\\
  Våghastighet:
  \[
    V_{quantum}=\frac{\hbar |k|}{2m}=\sqrt{\frac{E}{2m}}
  \]
  \[
    V_{classic}=\sqrt{\frac{2E}{m}}
  \]
  \begin{enumerate}
    \item $V_{quantum}=\frac{V_{classic}}{2}$
    \item \textbf{Normering?}\\
    $\Psi(x,t)=Ae^{i(kx-\frac{\hbar k^2}{2m}t)}$ går ej att normera.
    \textbf{Fri partikel kan ej beskrivas med stationärt tillstånd!}
  \end{enumerate}
  Bygg ett \textbf{vågpaket}:
  \[
    \Psi(x,t)=\frac{1}{\sqrt{2\pi}}\int_{-\infty}^{+\infty}\phi(k)e^{i(kx-\frac{\hbar k^2}{2m}t)}dk
  \]
  Denna är normerbar!\\
  Om $\Psi(x,0)$ känd $rightarrow\phi(k)$ [Plauchelet's theorem]:
  \[
    \phi(k)=\frac{1}{\sqrt{2\pi}}\int_{-\infty}^{+\infty}\Psi(x,0)e^{-ikx}dx
  \]
  alla $\omega(k)$:
  \begin{itemize}
    \item Taylor expansion ($k_0\rightarrow\omega(k)\approx\omega_0+\omega_0'(k-k_0)$)
    \item variabel $\omega(k)=\frac{\hbar k^2}{2m}$
  \end{itemize}
  \[
    \Psi(x,t)=\frac{1}{\sqrt{2\pi}}\int_{-\infty}^{+\infty}\phi(k)e^{i(kx-\omega t)}dk
  \]

\section{Potential barriär \textit{(potential step)}}
  Potetnialen $V(x)$ består utav en stegfunktion så att den går mot oändligheten åt ena hållet med värdet $0$ och sedan ett värde $V_0$ åt andra hållet.
  \[
    V(x) = V_0H(x) =
    \begin{cases}
      0, & x<0\\
      V_0, & x>0
    \end{cases}
  \]
  $H(x)$ är heaviside funktionen.\\

  \textbf{FALL I: Om $V=0$, $x<0$}
    \[
      \frac{d^2\Psi}{dx^2}+k^2\Psi,\quad k^2=\frac{2m}{\hbar^2}E
    \]
    Egenfunktion:
    \[
      \Psi(x)=Ae^{ikx}+Be^{-ikx}
    \]
    Om vågen kommer från vänster så kommer en reflekterande tillbaka

  \textbf{FALL II: Om $V=V_0$, $x>0$}
    \[
      \frac{d^2\Psi}{dx^2}+q^2\Psi,\quad q^2=\frac{2m}{\hbar^2}(E-V_0)
    \]
    Egenfunktion:
    \[
      \Psi(x)=Ce^{iqx}
    \]
    Här blir det ingen reflekterad våg.\\

    \begin{itemize}
      \item vågfunktionen $\Psi(x)$ är kontinuerlig
      \item $\frac{d\Psi}{dx}$ är kontinuerlig (Om $V$ är ändlig)
    \end{itemize}

    Vid randen mellan de två olika lösningarna matchar vi de två vågekvationerna.\\
    Vid $x=0$
    \[
      Ae^{ik0}+Be^{-ik0}=Ce^{iq0}\Longrightarrow A+B=C
    \]
    För stationära tillstånd i 1D: \textbf{Probability current density} \textit{(se problem 1.14)}:\\
    \textbf{FALL I}
    \[
      j=\frac{\hbar}{2im}(\Psi^*\frac{d\Psi}{dx}-\frac{d\Psi^*}{dx}\Psi)=\frac{\hbar}{2im}\bigg((A^*e^{-ikx}+B^*e^{ikx})(Ae^{ikx}-Be^{-ikx})-(-A^*ike^{-ikx}+B^*ike^{ikx})(Ae^{ikx}-Be^{-ikx})\bigg)
    \]
    \[
      j=\frac{\hbar k}{2m}(|A|^2-|B|^2)
    \]\\
    \textbf{FALL II}
    \[
      \Psi(x)=Ce^{iqx}
    \]
    \[
      j=\frac{\hbar q}{m}|C|^2
    \]
    Strömmen i (\textbf{I, II}) ska vara lika vid $x=0$:
    \[
      j_I=j_{II}
    \]
    Derivatan är kontinuerlig vid $x=0$
    \[
      \bigg(\frac{d\Psi}{dx}\bigg)_I=Aike^{ikx}-Bike^{-ikx}
    \]
    \[
      \bigg(\frac{d\Psi}{dx}\bigg)_{II}=Ciqe^{ikx}
    \]
    \textbf{DEF:} Reflexionskoefficient
    \[
      R=\frac{\frac{\hbar k}{m}|B|^2}{\frac{\hbar k}{m}|A|^2}=\frac{|B|^2}{|A|^2}
    \]
    \textbf{DEF:} Transmissionskoefficient
    \[
      T=\frac{q|C|^2}{k|A|^2}
    \]
    Uttrycka koeficienterna i endast $k$ och $q$:
    \[
      R=\frac{(k-q)^2}{(k+q)^2}
    \]
    \[
      T=\frac{4kq}{(k+q)^2}
    \]
    Beskrivet med energi och potential:
    \[
      R=\frac{(1-\sqrt{1-\frac{V_0}{E}})^2}{(1+\sqrt{1-\frac{V_0}{E}})^2}
    \]
    \[
      T=\frac{4\sqrt{1-\frac{V_0}{E}}}{(1+\sqrt{1-\frac{V_0}{E}})^2}
    \]
\section{Potential barriär}
  En potential barriär är en fyrkantspuls.
  \[
    V(x) = V_0(H(x+a)-H(x-a))
  \]
  Man kan dela upp det i tre fall, innan, i och efter barriären.\\

  \textbf{FALL II} i barriären:
  \[
    \kappa^2 = \frac{-2m(E-V_0)}{\hbar^2}
  \]
Lösningen: $\Psi_{II}=Ce^{-\kappa x}+De^{\kappa x}$\\

\textbf{FALL I} $\Psi_I=Ae^{ikx}+Be^{-ikx}$\\

\textbf{FALL III} $\Psi_{III}=Ee^{ikx}$\\

T ges utav:
\[
  |T|^2=\frac{2k\kappa}{(k^2+\kappa^2)^2\sinh^2{(2\kappa a)}+(2k\kappa)^2}
\]
Detta kallas \textit{Tunneling}


\end{document}
