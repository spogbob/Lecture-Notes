\documentclass{article}
\usepackage{amsmath}

\begin{document}
  \textbf{Från och med lektion 6!!!}
  \section{Potential som elastisk potentiell energi}
  I klassisk mekanik är elastisk potentiell energi $\frac{1}{2}kx^2$ men inte riktigt i QM.\\

  Går att lösa på två sätt:
  \begin{enumerate}
    \item Algebraisk metod
    \item Analytisk metod
  \end{enumerate}

  \subsection{Algebraisk metod}
  Tar ut algebraiskt med hjälp av TOSE stegoperatorerna $\hat{a}_+$ och $\hat{a}_-$. Stegoperatorerna är ej kommutativa så vi kan inte bara byta plats på dem.

  Vi kan ta fram en kommutator för lägesoperatorn och rörelsemängdsoperatorn $[\hat{x},\hat{p}]$:\\
  Vi använder oss av en testfunktion $f(x)$ för att kunna evaluera Stegoperatorerna.

  \begin{equation}
    \begin{split}
      [\hat{x},\hat{p}]f(x)=(\hat{x}\hat{p}-\hat{p}\hat{x})f(x)=-i\hbar\bigg(x\frac{d}{dx}f(x)-\frac{d}{dx}(xf(x))\bigg)=\\
      -i\hbar\bigg[x\frac{d}{dx}f(x)-\frac{dx}{dx}f(x)-x\frac{d}{dx}f(x)\bigg]=i\hbar f(x)
    \end{split}
  \end{equation}

  $[\hat{x},\hat{p}]=i\hbar$ kallas "canonical commutation relation"/"Osäkerhets sammbandet"\\

  Detta resultat appliceras på stegoperatorerna:
  \begin{equation}
    \hat{a}_-\hat{a_+}=\frac{1}{2\hbar m \omega}[\hat{p}^2+(m\omega\hat{x})^2]-\frac{i}{2\hbar}[\bar{x},\bar{p}]
  \end{equation}
\begin{equation}
  \hat{a}_-\hat{a_+}=\frac{1}{\hbar\omega}\hat{H}-\frac{i}{2\hbar}(i\hbar)=\frac{1}{\hbar\omega}\bar{H}+\frac{1}{2}
\end{equation}

\begin{equation}
  \hat{H}=\hbar(\hat{a}_-\hat{a}_+-\frac{1}{2})
\end{equation}
Jämföra med att byta plats på stegoperatorerna:
\begin{equation}
  \hat{a}_+\hat{a}_-=\frac{1}{\hbar\omega}\bar{H}-\frac{1}{2}
\end{equation}

\begin{equation}
  \hat{H}=\hbar(\hat{a}_+\hat{a}_-+\frac{1}{2})
\end{equation}

Vilker leder till att vi kan skriva hamiltonoperatorn $\hat{H}=\hbar\omega(\hat{a}_{\mp}\hat{a}_{\pm}\mp\frac{1}{2})$\\

Sätter in uttrycket i SE:
\begin{equation}
  \hbar\omega(\hat{a}_{\mp}\hat{a}_{\pm}\mp\frac{1}{2})\Psi=E\Psi
\end{equation}

Om vågfunktionen $\Psi$ är en egenfunktion med egenvärde $E$, vad är $\hat{a}_{\pm}\Psi$?

Från SE för harmonisk oscillator:\quad $\hat{H}\hat{a}_+\hat{a}_-+\frac{1}{2}=\hbar\omega(\hat{a}_+\hat{a}_-+\frac{1}{2})\hat{a}_+\Psi$

\begin{equation}
  \hat{H}(\hat{a}_+,\Psi)=\hbar\omega(\hat{a}_+\hat{a}_-\hat{a}_++\frac{1}{2}\hat{a}_+)\Psi=\hbar\omega\hat{a}_+(\hat{a}_-\hat{a}_++\frac{1}{2})\Psi
\end{equation}

För att kommutatorn $[\hat{a}_-,\hat{a}_+]$ ska vara kommutativt så måste den vara lika med $1$ detta leder till:

\begin{equation}
  \hat{H}(\hat{a}_+\Psi)=\hbar\omega\hat{a}_+(1+\hat{a}_+\hat{a}_-+\frac{1}{2})\Phi=\hat{a}_+(\hat{H}+\hbar\omega)\Psi=\hat{a}_+(E+\hbar\omega)\Psi
\end{equation}

Mellantermen är en konstant och kommutativ med de andra termerna:
\begin{equation}
  \hat{H}(\hat{a}_+\Psi)=(E+\hbar\omega)\hat{a}_+\Psi
\end{equation}

Detta betyder att om $\hat{H}\Psi=E\Psi$ så är $\hat{a}_+\Psi$ en egenfunktion med egenvärde $E+\hbar\omega$.\\
På samma sätt så ger det att $\hat{a}_-$ är en egenfunktion med egenvärde $E-\hbar\omega$.
\begin{equation}
  \hat{H}(\hat{a}_-\Psi)=(E-\hbar\omega)\hat{a}_-\Psi
\end{equation}

Detta gör så att om man multiplicerar vågfunktionen med steg-upp-operatorn $\hat{a}_+$ så ökar energin med $\hbar\omega$ för varje gång man gör det. Och på samma sätt för steg-ner-operatorn men denna kan inte gå oändligt långt ner utan det finns ett stopp nedåt på $E_0$.

\textbf{OBS:} $\hat{a}_{\pm}\Psi$ behöver inte vara normerade även om $\Psi$ är det.

Vi tar lägsta energinivån $E_0$, om vi applicerar $\hat{a}_-$: $\hat{a}_-\Psi_0=0$\\

I SE:
\begin{equation}
  \frac{d}{dx}\Psi_0+\frac{m\omega}{\hbar}x\Psi_0=0
\end{equation}

Om man integrerar detta får man:
\begin{equation}
  \ln{\Psi_0}=-\frac{m\omega}{\hbar}\frac{x^2}{2}+c
\end{equation}
vilket kan skrivas om till:
\begin{equation}
  \Phi_0=Ae^{-\frac{m\omega}{2\hbar}x^2}
\end{equation}

För allmänn lösning så behöver vi först normera $\Psi_0$
\begin{equation}
  1=\int_{-\infty}^{+\infty}|A|^2e^{-(m\omega x^2)/\hbar}dx=A^2\sqrt{\frac{\pi\hbar}{m\omega}}=1
\end{equation}
\begin{equation}
  A =\sqrt[4]{\frac{m\omega}{\pi\hbar}}
\end{equation}
Vilket ger:
\begin{equation}
  \Phi_0=\sqrt[4]{\frac{m\omega}{\pi\hbar}}e^{-\frac{m\omega}{2\hbar}x^2}
\end{equation}
Detta är stationära tillståndet för harmoniska oscillatorn för lägsta energi.\\

Hur ser $E_0$ ut?\\
Från SE: Beräkna energi för $\Psi_0$:
\begin{equation}
  \hat{H}=\hbar\omega(\hat{a}_+\hat{a}_-+\frac{1}{2})=\frac{1}{2}\hbar\omega\Psi_0=E_0\Psi_0
\end{equation}
\begin{equation}
  E_0=\frac{1}{2}\hbar\omega
\end{equation}

$\Psi_0$, $E_0$ är grundtillståndet vilket gör att de resterande är exiterade tillstånd och kan beskrivas med hjälp av att kliva uppåt med $\hat{a}_+$:\\

Energi för exiterade tillstånd:
\begin{equation}
  E_n=(n+\frac{1}{2})\hbar \omega
\end{equation}

vågfunktionen för exiterade tillstånd:
\begin{equation}
  \Phi_n(x)=A_n(\hat{a}_+)^n\Phi_0(x)
\end{equation}
Där $n$ är större eller lika med $0$.

Den generella bågfunktionen skulle då bli:
\begin{equation}
  \Psi(x,t)=\sum_{n=0}^{\infty}c_n\Psi_n(x,t)
\end{equation}

\textbf{Titta själv på sida 45-46 och räkna på det!}

\end{document}
